\documentclass{article}

\usepackage[utf8]{inputenc}
\usepackage[portuguese]{babel}
\usepackage{geometry}
\usepackage[svgnames]{xcolor}
\usepackage{tikz}
\usepackage{tikz-layers}
\usepackage{fancyhdr}


\geometry{a4paper, top=2cm, bottom=2cm, right=1.5cm, left=1cm}

\definecolor{facebook}{RGB}{59,89,152}

\usetikzlibrary{calendar,positioning,matrix}
\usetikzlibrary{arrows,shapes}
\usetikzlibrary{er}
\usetikzlibrary{decorations.markings, patterns}



\def\pgfcalendarmonthname#1{%
	\ifcase#1\or Janeiro\or Fevereiro\or Março\or Abril\or
	Maio\or Junho\or Julho\or Agosto\or
	Setembro\or Outubro\or Novembro\or Dezembro\fi
}


\pagestyle{fancy}
\renewcommand{\headrulewidth}{0pt}
\fancyhf{}

\fancyhead[L]{
	\begin{tikzpicture}[overlay, remember picture]
		\node[anchor=west,fill=facebook!20, font={\large\sc},text=facebook, rounded corners=1mm, text width=\linewidth](title) at ([xshift=-1.015\linewidth]diaExtenso.east) {Dia};
	\end{tikzpicture}
}


\newcommand{\marcarH}[4][fill=Tan!20]{
	\begin{tikzpicture}[overlay, remember picture]
	\begin{scope}[on glass layer]
	\pgfmathtruncatemacro\marcai{2*#2-13}
	\pgfmathtruncatemacro\marcaii{2*#2-13+#3-1}
	\pgfmathsetmacro\des{(-2*#2+13+\marcai)*6}
	\draw[draw,rounded corners=1pt,#1] ([xshift=.1mm,yshift=\des mm]diaMatriz-\marcai-1.north west) rectangle ([xshift=-.1mm,yshift=\des mm]diaMatriz-\marcaii-1.south east);
	\path ([xshift=.1mm,yshift=\des mm]diaMatriz-\marcai-1.north west) to node[midway, text width=6cm,anchor=west]{#4} ([xshift=.1mm,yshift=\des mm]diaMatriz-\marcaii-1.south west);
	\end{scope}
	\end{tikzpicture}}


\newcommand{\marcarHd}[4][fill=CadetBlue!20]{
	\begin{tikzpicture}[overlay, remember picture]
	\begin{scope}[on glass layer]
	\pgfmathtruncatemacro\marcai{2*#2-13}
	\pgfmathtruncatemacro\marcaii{2*#2-13+#3-1}
	\pgfmathsetmacro\des{(-2*#2+13+\marcai)*6}
	\draw[draw,rounded corners=1pt,#1] ([xshift=0mm,yshift=\des mm]diaMatriz-\marcai-1.north) rectangle ([xshift=-.1mm,yshift=\des mm]diaMatriz-\marcaii-1.south east);
	\path ([xshift=.1mm,yshift=\des mm]diaMatriz-\marcai-1.north) to node[midway,opacity=1, text width=4cm,anchor=west]{#4} ([xshift=.1mm,yshift=\des mm]diaMatriz-\marcaii-1.south);
	\end{scope}
	\end{tikzpicture}}


\newcommand{\marcarHe}[4][fill=Chocolate!20]{
	\begin{tikzpicture}[overlay, remember picture]
	\begin{scope}[on glass layer]
	\pgfmathtruncatemacro\marcai{2*#2-13}
	\pgfmathtruncatemacro\marcaii{2*#2-13+#3-1}
	\pgfmathsetmacro\des{(-2*#2+13+\marcai)*6}
	\draw[draw,rounded corners=1pt,#1] ([xshift=.1mm,yshift=\des mm]diaMatriz-\marcai-1.north west) rectangle ([xshift=-0mm,yshift=\des mm]diaMatriz-\marcaii-1.south);
	\path ([xshift=.1mm,yshift=\des mm]diaMatriz-\marcai-1.north west) to node[midway,opacity=1, text width=4cm,anchor=west]{#4} ([xshift=.1mm,yshift=\des mm]diaMatriz-\marcaii-1.south west);
	\end{scope}
	\end{tikzpicture}}


%%----------------%%
%%  Dia a marcar  %%
%%----------------%%
\def\ano{2020}
\def\mes{12}
\def\diaMarcar{22}
%%----------------%%



\begin{document}
		\begin{tikzpicture}[remember picture,overlay]
	
		\node[anchor=south east,yshift=9.5cm, xshift=2cm] (calendario) at (current page.south){
			\begin{tikzpicture}[remember picture, overlay,scale=1.5]
				\calendar(MesI)[%
				dates=\ano-\mes-01 to \ano-\mes-last,
				day yshift=4mm,
				day xshift=8mm,
				week list,
				%every day/.style={anchor=base},
				day text={\%d=},rounded corners=0,anchor=base,text height=1ex,text depth=-0.5ex,
				month label above left,
				month text={
					\begin{tikzpicture}[remember picture, overlay]
					\node[anchor=west,yshift=.4cm] (nomeMes){\%mt, \%y-};
					\node[anchor=north east,xshift=-12mm, yshift=-5mm,text=facebook] (diaExtenso) at (current page.north east){\diaMarcar{} de \%mt, \ano};
					\end{tikzpicture}	
				},
				]
				if (Saturday) [black!40]
				if (Sunday) [red!40]
				if (equals=\ano-\mes-\diaMarcar)[blue!80]
				if (equals=\ano-\mes-\diaMarcar){\node [anchor=south east,scale=1.4,draw,circle,xshift=-1.5pt,yshift=-1pt] {};}
				if (equals=\ano-10-5)[red!40, day text=F]
				if (equals=\ano-11-1)[red!40, day text=F]
				if (equals=\ano-12-1)[red!40, day text=F]
				if (equals=\ano-12-8)[red!40, day text=F]
				if (equals=\ano-12-25)[red!40, day text=N]
				if (equals=\ano-1-1)[red!40, day text=F]
				if (equals=\ano-4-25)[red!40, day text=F]
				if (equals=\ano-5-1)[red!40, day text=F]
				if (equals=\ano-5-10)[red!40, day text=F]
				if (equals=\ano-8-15)[red!40, day text=F];
			\end{tikzpicture}
			};
		
			
			\node[anchor=north,yshift=-5cm,] (calendarioSeg) at (calendario.south){
				\pgfmathtruncatemacro\messeg{\mes+1}
				\begin{tikzpicture}[remember picture, overlay,scale=1.5]
				\calendar(MesII)[%
				dates=\ano-\messeg-01 to \ano-\messeg-last,
				day yshift=4mm,
				day xshift=8mm,
				week list,
				month label above left,
				month text={
					\begin{tikzpicture}[remember picture, overlay]
					\node[anchor=west,yshift=.4cm] (nomeMesSeg){{\%mt, \%y-}};
					\end{tikzpicture}	
				},
				]
				if (Saturday) [black!40]
				if (Sunday) [red!40]
				if (equals=\ano-\mes-\diaMarcar)[blue!40]
				if (equals=\ano-10-5)[red!40, day text=F]
				if (equals=\ano-11-1)[red!40, day text=F]
				if (equals=\ano-12-1)[red!40, day text=F]
				if (equals=\ano-12-8)[red!40, day text=F]
				if (equals=\ano-12-25)[red!40, day text=N]
				if (equals=\ano-1-1)[red!40, day text=F]
				if (equals=\ano-4-25)[red!40, day text=F]
				if (equals=\ano-5-1)[red!40, day text=F]
				if (equals=\ano-5-10)[red!40, day text=F]
				if (equals=\ano-8-15)[red!40, day text=F];
			\end{tikzpicture}
		};
		
		\foreach\i in{nomeMes, nomeMesSeg}{
			\node[anchor=west] (DiasSemana) at ([yshift=-4mm,xshift=-4mm]\i.west){
				\begin{tikzpicture}[remember picture, overlay,scale=1.5]
					\matrix[matrix of nodes,
					column 6/.style={text=black!40},
					column 7/.style={text=red!40},
					nodes={minimum width=12mm, minimum height=1em},
					ampersand replacement=\&] (Dias){%
						S \& T \& Q \& Q \& S \& S \& D\\};
				\end{tikzpicture}
			};
		}
	\end{tikzpicture}

	\begin{tikzpicture}[overlay, remember picture]
		\node[anchor=north west,xshift=1cm, yshift=-2cm] (ponteado) at (current page.north west){
			\begin{tikzpicture}
			\foreach \i in {.5,1,...,10}{
				\foreach \j in {.5,1,...,27}{
					\draw (\i,\j) circle (.2pt);
				}
			}
			\end{tikzpicture}
		};
	\end{tikzpicture}

	
	\begin{tikzpicture}[overlay,remember picture]
	\begin{scope}[on glass layer]
		\node[anchor=north east,xshift=-1cm, yshift=-1cm] (Dia) at (current page.north east){
			\begin{tikzpicture}[remember picture, overlay]
				\matrix[matrix of nodes,
						nodes in empty cells,
						nodes={minimum width=8cm, minimum height=6mm,draw, rounded corners=1pt}] (diaMatriz) {
						 \\
						 \\ 
						 \\ 
						 \\ 
						 \\ 
						 \\ 
						 \\ 
						 \\ 
						 \\ 
						 \\ 
						 \\ 
						 \\ 
						 \\ 
						 \\ 
						 \\ 
						 \\ 
						 \\ 
						 \\ 
						 \\ 
						 \\ 
						 \\ 
						 \\ 
						 \\ 
						 \\ 
						 \\ 
						 \\ 
						 \\ };
			\end{tikzpicture}
		};	
		\foreach \i in {1,3,...,27}{
			\pgfmathtruncatemacro\j{(\i+13)/2}:
			\node[anchor=east, font=\small] (time) at (diaMatriz-\i-1.north west){\j};
			\node[anchor=east, font=\scriptsize] (time) at (diaMatriz-\i-1.south west){:30};
		}
	\end{scope}
	
	\end{tikzpicture}
	
	
	% Temos 3 comandos que funcionam da mesma maneira:
	% \marcarH[opt. arg1]{arg2}{arg3}{arg4}
	% \marcarHd[opt. arg1]{arg2}{arg3}{arg4}
	% \marcarHe[opt. arg1]{arg2}{arg3}{arg4}
	% significam "marcar Hora" e depois temos "d" de direita ou "e" de esquerda.
	%
	% No opt.arg1 colocamos configurações ao formato da marcação
	% arg2 a hora de começo, se for 8:30, colocamos 8.5 (podemos calcular este valor)
	% arg3 quantos blocos ocupa, cada bloco tem 30 minutos
	% arg4 o que escrever no bloco
	
	\pgfcalendarifdate{\ano-\mes-\diaMarcar}{Monday}{
	\marcarH{7}{1}{Up Up!}
	\marcarHd{8.5}{2}{Aula 1}
	\marcarHe{9.5}{2}{Aula 2}
	\marcarH[top color=facebook!20, fill opacity=.95]{11}{3}{Aula 3}
	\marcarH[pattern=north west lines, opacity=.2]{14.5}{5}{Aula 4}
	}
	{}
	
	\pgfcalendarifdate{\ano-\mes-\diaMarcar}{Tuesday}{
		\marcarH{7}{1}{Up Up!}
		\marcarHd{9}{2}{Aula 1}
		\marcarHe{10.33}{2}{Aula 2}
		\marcarH[top color=facebook!20, fill opacity=.95]{14}{3}{Aula 3}
		\marcarH[pattern=north west lines, opacity=.2]{17}{5}{Aula 4}
	}
	{}
	
	
	% É só copiar esta instrução para completar outros dias e depois aparecerá o dia correctamente preenchido.

\end{document}