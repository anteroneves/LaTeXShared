\documentclass{standalone}


\usepackage{tkz-euclide}
\usepackage{xifthen}

\begin{document}
	
	\begin{tikzpicture}[scale=.2]
		
	\def\Ntermos{5} % Controla o número de termos a apresentar
	\foreach \k in {1,...,\Ntermos}{ % Vamos definir o que acontece para cada termo
		\pgfmathtruncatemacro{\dist}{(5+\k)^2}; % define uma função para a distância deste o primeiro até ao termo \k.
		\begin{scope}[xshift=\dist cm ] % Este xshift desloca a construção do termo \k de acordo com \dist definida na linha anterior (distância entre bolas iniciais de cada termo)
			\def\raio{1.2} % define o raio do círculo
			\def\Termo{\k} % define até que linha vamos desenhar (chamei termo porque o número de linhas é igual à ordem do termo)	
			\foreach \j in {1,...,\Termo}{
				\tkzDefPoint(120:{2*\j*\raio}){O\j} % define onde vai ficar o centro O da primeira bola de cada linha, dependendo da linha j e do raio. Dependendo do valor do ângulo podemos fazer muitas figuras diferentes.
				\begin{scope}[shift=(O\j)] % o shift muda a posição do centro do referencial para o O\j definido anteriormente.
					\foreach \i in {1,...,\j}{
						\begin{scope}[xshift=2*\raio*\i cm] % vai avancando a origem do refencial para construir a linha
							\tkzDefPoint(0,0){A\j\i} % define o centro do círculo, tenho A11 vai ser o nome de cada bola inicial, fiz assim para depois ter uma referência para colocar a legenda no termo automaticamente.
							\tkzDrawCircle[R](A\j\i,\raio cm)
							\ifthenelse{\i=1 \OR \i=\j}%testa se estamos na primeira ou na última bola
							{ 
								\shade[ball color=red] (A\j\i) circle [radius=\raio cm]; % faz isto se for a primeira ou última bola
							}
							{
								\shade[ball color=blue] (A\j\i) circle [radius=\raio cm]; % faz isto se não for a primeira ou a última bola.
							}
						\end{scope}
					};
				\end{scope}
			};
		\end{scope}
	\tkzLabelPoint[below=5mm](A11){\(\k\).º termo} % coloca a legenda no termo
	};

	\end{tikzpicture}

\end{document}