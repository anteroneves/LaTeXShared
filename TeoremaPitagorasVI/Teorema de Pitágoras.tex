\documentclass[12pt]{article}

\usepackage{geometry}
\usepackage{tkz-euclide}
\usepackage{fancyhdr}
\usepackage{graphicx}
\usepackage{fouriernc}

\def\facebookSquare{900pt}
\def\margem{75pt}


\geometry{
	paperheight=\facebookSquare, 
	paperwidth=\facebookSquare,
	textwidth=750pt}

\definecolor{facebook}{RGB}{59,89,152}
\definecolor{youtube}{HTML}{90030c}

\begin{document}
	
	\thispagestyle{empty}
	
	
	
	\begin{tikzpicture}[remember picture, overlay]
		\fill[facebook!70] (current page.north west) rectangle (current page.south east);
		\fill[white] ([xshift=\margem, yshift=-\margem]current page.north west) rectangle ([xshift=-\margem,yshift=\margem]current page.south east);
		
		\node[anchor=center] (figura) at (current page.center) {
			\begin{tikzpicture}[scale=2]
				
				% Definição dos três vértices do triângulo rectângulo. Desenho do triângulo.
				\tkzDefPoints{0/0/A, 4/0/B, 0/3/C}
				\tkzDrawPolygon(A,B,C)
				
				% Definição dos três quadrados a partir dos lados do triângulo. Desenho dos quadrados.
				
				% Quadrado cateto 1
				\tkzDefSquare(B,A)\tkzGetPoints{D}{E}
				\tkzDrawPolygon(B,A,D,E)
				
				% Quadrado cateto 2
				\tkzDefSquare(A,C)\tkzGetPoints{F}{G}
				\tkzDrawPolygon(A,C,F,G)
				
				% Quadrado hipotenusa
				\tkzDefSquare(C,B)\tkzGetPoints{H}{I}
				\tkzDrawPolygon(C,B,H,I)
				
				% Definição de pontos para divisão do quadrado da hipotenusa.
				
				% Quadrado central (Q)
				\tkzInterLL(E,B)(F,C)\tkzGetPoint{Q1}
				\tkzDefPointBy[projection=onto B--E](H)\tkzGetPoint{Q2}
				\tkzDefSquare(Q1,Q2) \tkzGetPoints{Q3}{Q4}
				
				% Restante divisão
				\tkzInterLL(G,C)(I,Q3)\tkzGetPoint{I'}
				\tkzDefLine[parallel=through H](G,C)\tkzGetPoint{H'}
				\tkzInterLL(B,E)(H,H')\tkzGetPoint{B'}
				
				
				% Definição de pontos para divisão do cateto 1
				
				\tkzInterLL(B,H)(D,E)\tkzGetPoint{D'}
				\tkzDefSquare(D,D')\tkzGetPoints{D''}{D'''}
				
				
				% Desenho dos diferentes polígonos
				
				% Desenho dos polígonos do quadrado da hipotenusa
				\tkzDrawPolygon[fill=facebook, fill opacity=.6](Q1,Q2,Q3,Q4)
				\tkzDrawPolygon[fill=facebook, fill opacity=.3](Q1,C,B)
				\tkzDrawPolygon[fill=facebook, fill opacity=.3](I,Q3,H)
				\tkzDrawPolygon[fill=facebook, fill opacity=.1](B,B',H)
				\tkzDrawPolygon[fill=facebook, fill opacity=.1](I,I',C)
				\tkzDrawPolygon[fill=youtube, fill opacity=.2](Q4,C,I')
				\tkzDrawPolygon[fill=youtube, fill opacity=.2](Q2,H,B')
				
				% Desenho dos polígonos do cateto 1
				\tkzDrawPolygon[fill=facebook, fill opacity=.6](D,D',D'',D''')
				\tkzDrawPolygon[fill=facebook, fill opacity=.3](A,B,D''')
				\tkzDrawPolygon[fill=facebook, fill opacity=.3](B,E,D')
				\tkzDrawPolygon[fill=facebook, fill opacity=.1,rounded corners=.1pt](B,D',D'')
				\tkzDrawPolygon[fill=facebook, fill opacity=.1,rounded corners=.1pt](B,D'',D''')
				
				%Desenho dos polígonos do cateto 2
				\tkzDrawPolygon[fill=youtube, fill opacity=.2](A,C,G)
				\tkzDrawPolygon[fill=youtube, fill opacity=.2](C,F,G)
				
				%\tkzLabelPoints(A,...,I)
			\end{tikzpicture}
		};
	
		\node[anchor=north, scale=2.5] (titulo) at ([yshift=-\margem]current page.north){Teorema de Pitágoras};
		
		\node[anchor=south west,scale=1, text width=6cm] (info) at ([yshift=\margem, xshift=\margem]current page.south west) {
			Demonstração por: Liu Hui
			
			Fonte: Proofs without words II
			};
		
;	\end{tikzpicture}
	
	
	

\end{document}