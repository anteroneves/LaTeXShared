\documentclass{article}

\usepackage[portuguese]{babel}
\usepackage{fouriernc}
\usepackage{tkz-fct}
\usepackage{tkz-euclide}
\usepackage{pythontex}
\usepackage{geometry}

\geometry{top=2cm, bottom= 2cm, left=2cm, right=2cm}

\title{Distância de um ponto a uma reta}
\author{Antero Neves}
\begin{document}


\begin{pythontexcustomcode}{sympy}
from sympy import *

x = Symbol('x')	

def ilustraDist(P,f,Janela):	
	Q = Point(Janela[0],f.subs(x,Janela[0]))
	R = Point(Janela[1],f.subs(x,Janela[1]))
	r = Line(Q,R)
	s = r.perpendicular_line(P)
	I = r.intersection(s)
	S = Point(Janela[0],Janela[3])
	T = Point(Janela[1],Janela[3])
	t = Line(S,T)
	U = r.intersection(t)
	print(r"\begin{tikzpicture}")
	print(r"\tkzInit[xmin=%s,xmax=%s,ymin=%s,ymax=%s]"%(Janela[0],Janela[1],Janela[2],Janela[3]))
	print(r"\tkzClip[space=1pt]")
	print(r"\tkzDrawXY[>=stealth',noticks]")
	print(r"\tkzDefPoint({%s},{%s}){P}"% (P[0],P[1]))
	print(r"\tkzDefPoints{{%s}/0/B, 0/{%s}/C}"%(P[0],P[1]))
	print(r"\coordinate (Q) at (%f,%f);"%(Q[0],Q[1]))
	print(r"\coordinate (R) at (%f,%f);"%(R[0],R[1]))
	print(r"\coordinate (I) at (%f,%f);"%(I[0][0],I[0][1]))
	print(r"\draw[thick] (Q) -- (R);")
	print(r"\tkzDrawLine[opacity=.5, add=2cm and 2cm, dotted](P,I)")
	print(r"\tkzDrawSegment[red](P,I)")
	print(r"\tkzDrawSegments[dashed](B,P C,P)")
	print(r"\tkzDrawPoints[fill=white](P,I)")
	print(r"\tkzLabelPoints[above left](P)")
	print(r"\node[anchor=north, fill=white, opacity=.7,text opacity=1,outer sep=1pt] (xP) at (%f,0){\(%s\)};"%(P[0],latex(P[0])))
	print(r"\node[anchor=east, fill=white, opacity=.7,text opacity=1,outer sep=1pt] (yP) at (0,%f){\(%s\)};"%(P[1],latex(P[1])))
	print(r"\node[anchor=south west, fill=white, opacity=.7,text opacity=1] (r) at (%f,%f){\(r\)};"%(U[0][0],U[0][1]))
	print(r"\node[anchor=north east, fill=white, opacity=.7,text opacity=1](Ponto) at (%f,%f){Ponto:\(\left(%s,%s\right)\)};"%(Janela[1],Janela[3],latex(P[0]),latex(P[1])))
	print(r"\node[anchor=north east, fill=white, opacity=.7,text opacity=1](reta) at (Ponto.south east) {\(\displaystyle r:y=%s\)};"%(latex(simplify(f))))
	print(r"\node[anchor=north east, fill=white, opacity=.7,text opacity=1,text=red](dist) at (reta.south east) {\(\displaystyle d=%s\)};"%(latex(simplify(r.distance(P)))))
	print(r"\end{tikzpicture}")
\end{pythontexcustomcode}

\maketitle

Instrução: \verb|ilustraDist|

Requisitos: PythonTeX

Objectivo: Calcular a distância de um ponto a uma reta e fazer a ilustração.

Parâmetros: Ponto: \verb|(x,y)|, equação reduzida da reta: \verb|m*x+b|, janela de visualização: \verb|(xmin, xmax, ymin, ymax)|

Exemplo: \verb|\sympyc{ilustraDist((-1,4),3*x+1,(-2,4,-1,4.5))}|

\vfill

\sympyc{ilustraDist((-1,4),3*x+1,(-2,4,-1,4.5))}

\vfill

\sympyc{ilustraDist((-sqrt(5),1),-1/3*x+1.5,(-3,7,-1,2.5))}


\end{document}