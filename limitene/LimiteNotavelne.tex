\documentclass[12pt]{article}

\usepackage{geometry}
\usepackage{tkz-euclide,tkz-fct}
\usepackage{fancyhdr}
\usepackage{graphicx}
\usepackage{fouriernc}
\usepackage{amsmath, amssymb}
\usepackage{witharrows}

\def\facebookSquare{900pt}
\def\margem{75pt}


\geometry{
	paperheight=\facebookSquare, 
	paperwidth=\facebookSquare,
	textwidth=750pt}

\definecolor{facebook}{RGB}{59,89,152}
\definecolor{youtube}{HTML}{90030c}

\begin{document}
	
	\thispagestyle{empty}
	
	
	
	\begin{tikzpicture}[remember picture, overlay]
		\fill[facebook!70] (current page.north west) rectangle (current page.south east);
		\fill[white] ([xshift=\margem, yshift=-\margem]current page.north west) rectangle ([xshift=-\margem,yshift=\margem]current page.south east);
		
		\node[anchor=north west] (figura1) at ([xshift=\margem + 10pt, yshift=-\margem - 75pt]current page.north west) {
			\begin{tikzpicture}[scale=2.5]
				\tkzInit[xmin=-1, xmax=2, ymin=-1, ymax=2]
				\tkzDrawX[noticks, label=\(x\)]
				\tkzDrawY[noticks, label=\(y\)]
				
				\begin{scope}
					\tkzInit[xmin=-1, xmax=3, ymin=-1, ymax=2.3]
					\tkzFct[domain=0:2.5,thick]{1/x}
					\tkzDrawArea[color=youtube,fill opacity=.2, domain=1:2, draw=black]
					\tkzDefPoint[label=below:\(1\)](1,0){A}
					\tkzDefPoint[label=below:\(t\)](2,0){B}
					\tkzDefPointByFct(0.6)
					\tkzText[above right](tkzPointResult){\(y=\dfrac{1}{x}\)}
				\end{scope}
			
				\node[anchor=center] (texto) at (1.5,0.3) {\(\ln(t),\, t>1\)};
			\end{tikzpicture}
		};
	
		
		\node[anchor=north west] (figura2) at (figura1.east) {
			\begin{tikzpicture}[scale=5, every node/.style={scale=1.5}]
				\tkzInit[xmin=-.1, xmax=2, ymin=-.5, ymax=2]
				\tkzDrawX[noticks, label=\(x\)]
				\tkzDrawY[noticks, label=\(y\)]
				
				
				
				\begin{scope}
					
					\def\xP{1.8}
					
					\tkzInit[xmin=-.1, xmax=3, ymin=-.5, ymax=2.3]
					\tkzFct[domain=0:2.5,thick]{1/x}
					\tkzDrawArea[color=facebook,fill opacity=.2, domain=1:\xP, draw=black]
					
					\tkzDefPoint(1,1){A}
					\tkzDefPoint(1,0){xA}
					\tkzDefPoint[label=left:\(1\)](0,1){yA}
					\tkzDefPoint[label=below:\(\vphantom{\dfrac{1}{n}} 1\)](1,0){xA}
					
					
					\tkzDefPointByFct(\xP)\tkzGetPoint{B}
					\tkzDefPoint[label=below:\(1+\dfrac{1}{n}\)](\xP,0){xB}
					\tkzDefPoint[label=left:\(\dfrac{n}{n+1}\)](0,1/\xP){yB}
					
					\tkzDrawSegment[dashed](A,yA)
					
					\tkzDrawRectangle[pattern=north west lines,opacity=.2](A,B)
					\tkzDrawRectangle[pattern=north east lines,opacity=.2](A,xB)
					
					\tkzDrawRectangle[thick](A,B)
					\tkzDrawRectangle[thick](A,xB)
					
					\tkzDrawSegment[dashed](B,yB)
					
					\tkzDefPointByFct(0.6)
					\tkzText[above right](tkzPointResult){\(y=\dfrac{1}{x}\)}
				\end{scope}
			\end{tikzpicture}
		};
		
		
		\node[anchor=north,scale=1.5](demonstracao) at ([yshift=-1cm]figura1.south){
			\begin{DispWithArrows*}[format=cCcCc,interline=5mm]
			\dfrac{1}{n}\cdot \dfrac{n}{n+1} &\leq& \ln\left(1+\dfrac{1}{n}\right)&\leq& \dfrac{1}{n}\cdot 1\\
			\dfrac{n}{n+1} &\leq& n\cdot \ln\left(1+\dfrac{1}{n}\right) &\leq& 1\\
			\dfrac{n}{n+1} &\leq&\ln\left(1+\dfrac{1}{n}\right)^n &\leq& 1
			\end{DispWithArrows*}
			};
		
		\node[anchor=north,scale=1.5](conclusao) at ([yshift=-1cm]demonstracao.south){\(\therefore\, \lim \left[\ln\left(1+\dfrac{1}{n}\right)^n\right] = 1 \)};
		
		\node[anchor=north, scale=2.5, text centered] (titulo) at ([yshift=-\margem]current page.north){
		Por falar em
		\(\lim\left(1+\dfrac{1}{n}\right)^n = e\)};
		
		\node[anchor=south west,scale=1, text width=10cm] (info) at ([yshift=\margem, xshift=\margem]current page.south west) {
			Fonte: Proofs without words II - Roger B. Nelsen
			};
	

	\end{tikzpicture}
	
	
	

\end{document}
